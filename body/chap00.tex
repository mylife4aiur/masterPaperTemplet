% !TEX TS-program = XeLaTeX
% !TEX encoding = UTF-8 Unicode

\chapter*{\hfill 引  言 \hfill}
\addcontentsline{toc}{chapter}{引  言}
\label{chap00}



\section{Windows操作系统}

强烈建议下载最新版本的C\TeX{}。最新版本号为v2.9.2.164。建议选择完整版本
下载,因为本硕士论文模板使用的某些宏包比较新。不然可能会造成编译错误。

\subsection{编译运行}

需要注意的是,在WinEdit中必须在每个tex文件的开始添加如下的两行:
\begin{lstlisting}
  % !TEX TS-program = XeLaTeX
  % !TEX encoding = UTF-8 Unicode
\end{lstlisting}
否则文件会变成乱码。

以本模版为例,在Windows下的编译过程是这样的:
\begin{enumerate}
\item 打开main.tex文件;
\item 先点击WinEdt工具栏上的\XeLaTeX{}按钮(可能在Acrobat Reader按钮的下拉菜单
  中);
\item 再点击WinEdt工具栏上的Bib\TeX{}按钮;
\item 再点击WinEdt工具栏上的\XeLaTeX{}按钮两到三遍;
\item 最后点击WinEdt工具栏上的Acrobat Reader按钮就可以看到输出的PDF文档了。
\end{enumerate}


\section{Linux操作系统(以Ubuntu为例)}
First things first,首先的工作是安装一个合适的\XeTeX{}编译系统。这个问题
并不难解决,现在主流的\LaTeX{}编译系统均已经包含了对\XeTeX{}的支持(包
括xeCJK中文宏包),并不需要自己额外再进行安装。在Linux下推荐使
用\TeX{}Live,目前最新版本为\TeX{}Live 2011。下面以在Ubuntu下的本地安装为
例,简要的说明\TeX{}Live的安装及配置过程,高玩们请主动绕行:
\begin{enumerate}
\item 下载\TeX{}live 2011镜像,点
  击\href{http://ftp.ctex.org/mirrors/CTAN/systems/texlive/Images/}{这里}进
  入下载列表。如果你有检查文件完整性的习惯的话,这个列表还提供
  了md5和sha256校验值;
\item 安装perl-tk包,以便使用图形界面进行安装。在终端中输入命
  令\texttt{\footnotesize sudo apt-get install perl-tk};
\item 挂载下载好的iso镜像,\texttt{\footnotesize sudo mkdir
    /mnt/texlive}(在~{/mnt}~下创建texlive文件夹
  ),\texttt{\footnotesize sudo mount -o loop texlive2011.iso
    /mnt/texlive}(挂载texlive2011.iso)。进入~/mnt/texlive~目录,输入命
  令~\texttt{\footnotesize sudo ./install-tl -gui}~之后出现图形界面。之后
  的操作就比较简单了,可以去掉不用的语言包以节省磁盘空间,注意选择最后一
  项Create symlinks in system directories,让安装程序自动创建语法链接。确
  定安装,耐心等待进度条到头;
\item 配置环境变量。在终端中输入~\texttt{\footnotesize sudo gedit
    /etc/bash.bashrc},在此文件末尾添加
  \begin{lstlisting}
    PATH=/usr/local/texlive/2011/bin/i386-linux:$PATH; export PATH
    MANPATH=/usr/local/texlive/2011/texmf/doc/man:$MANPATH; export
    MANPATH
    INFOPATH=/usr/local/texlive/2011/texmf/doc/info:$INFOPATH;
    export INFOPATH
  \end{lstlisting}
  % $ 
  在~{/etc/manpath.config}~文件的~\texttt{\footnotesize\# set up PATH to
    MANPATH mapping}~这行下面的列表后增加一条:
  \begin{lstlisting}
    MANPATH_MAP /usr/local/texlive/2011/bin/i386-linux
    /usr/local/texlive/2011/texmf/doc/man
  \end{lstlisting}
  在~{/etc/manpath.config}~文件的~\texttt{\footnotesize\# set up PATH to
    MANPATH mapping}~这行下面的列表后增加一条:
  \begin{lstlisting}
    MANPATH_MAP /usr/local/texlive/2011/bin/i386-linux
    /usr/local/texlive/2011/texmf/doc/man
  \end{lstlisting}
\end{enumerate}
至此安装过程结束。

以上\TeX{}Live安装过程摘自某位筒子的博客文摘,原始链接位于wordpress空间,
访问有问题,不
过
\href{http://hi.baidu.com/skubuntu/blog/item/89e8de2f73a465e08a1399a3.html}{
  百度空间}有转载,虽然百度搜不着什么玩意。

接下来我们需要安装一套中文字体,你可以使用Windows下的方正、华文或者中易字
体,但要注意选择的字体最好是包含宋体、黑体、楷体和仿宋的完整套装。不过由
于这些字体在PDF浏览器中的显示效果并不好,所以建议选用Adobe的中文字体。安
装及配置过程如下:
\begin{enumerate}
\item 下载Adobe中文字体,点
  击
  \href{http://forum.ubuntu.org.cn/viewtopic.php?f=35&t=180987&start=0}{
    这里}进入下载页面;
\item 将下载的字体拷至~{/usr/share/fonts/truetype/adobe}~目录,如果没有请
  以管理员身份新建;
\item 刷新字体缓存,在终端中输入~\texttt{\footnotesize sudo fc-cache -fv }。这时,你可以通过~\texttt{\footnotesize fc-list :lang=zh-cn |sort}~命令来查看字体是否安装成功,注意fc-list后有个空格;
\item 你可能还需要在终端中运行~\texttt{\footnotesize sudo apt-get
    install poppler-data cmap-adobe-cns1 cmap-adobe-gb1}命令来解决Adobe中
  文字体在PDF文件中不显示的情况。
\end{enumerate}
这样,我们就配置好了中文字体,当然这没什么特别的,网上教程一大把。

之后我们需要一个类似于WinEdt的集成编译环境。在Ubuntu软件中心中,我们能很
容易的安装\TeX{}maker和\TeX{}works,两者功能差不多,\TeX{}maker更强大一些。
当然,你也可以自己配置VIM下的\LaTeX{}编译环境,在此就不赘述了。

\subsection{编译运行}


在安装并配置好编译环境之后,接下来的工作就是如何编译\XeLaTeX{}文件,生成
所需的PDF文档了。以本模版为例,在\TeX{}works编译过程是这样的:
\begin{enumerate}
\item 打开main.tex文件;
\item 将工具栏上的编译命令切换至\XeLaTeX{}后,点击运行;
\item 再将工具栏上的编译命令切换至Bib\TeX{}后,点击运行;
\item 再将工具栏上的编译命令切换至\XeLaTeX{}后,点击运行,这里需要运行两
  到三遍;
\item 如果编译没有错误,就可以看到输出的PDF文件了。
\end{enumerate}

对于\TeX{}maker,首先需要在【选项】【配置\TeX{}maker】【命令】中将第一行
的latex改成xelatex,之后用\LaTeX{}作为\XeLaTeX{}命令执行即可,其他的和上
面类似。

\section{XeTeX简介}
\XeTeX{}(英文发音为"zee-\TeX{}")是一种使用Unicode的\TeX{}排版引擎,并支
持一些现代字体技术,例如OpenType。其作者和维护者是Jonathan Kew,并以X11自
由软件许可证发布。

虽然\XeTeX{}最初只是为Mac OS X所开发,但它现在在各主要平台上都可以运作。
它原生的支持Unicode,并默认其输入文件为UTF-8编码。\XeTeX{}可以在不进行额
外配置的情况下直接使用操作系统中安装的字体,因此可以直接利
用OpenType,Graphite中的高级特性,例如额外的字形,花体,合字,可变的文本
粗细等等。\XeTeX{}提供了对OpenType中本地排版约定(locl标签)的支持,也允
许向字体传递OpenType的元标签。它亦支持使用包含特殊数学字符的Unicode字体排
版数学公式,例如使用Cambria Math或Asana Math字体代替传统的\TeX{}字体。


\subsection{历史}
2004年4月,发布了\XeTeX{}的第一个版本,这个版本只支持Mac OS X,并包括了内
建的ATT和Unicode支持。2005年,加入了对OpenType的支持。
在2006年Bacho\TeX{}期间,发布了第一个支持Linux的版本,并在数月后由Akira
Kakuto移植到了Microsoft Windows上,其跨平台版本最终包含在\TeX{}Live
2007中。另外,从2.7版开始,MiK\TeX{}也包含了\XeTeX{}。

作为\TeX{}Live的一部分,\XeTeX{}支持大多数
为\LaTeX{},OpenType,TrueType和PostScript字体开发的宏包,而无需特别的安
装和设定。


\section{引言内容要求}
以下给出研究生院对引言内容的要求,格式的要求已经嵌入到本模版中:
\begin{enumerate}
\item 引言包含的内容有说明论文的主题和选题的范围、对本论文研究主要范围内已有文献的评述以及
  说明本论文所要解决的问题;
\item 注意不要与摘要内容雷同;
\item 建议与相关历史回顾、前人工作的文献评论、理论分析等相结合,如果引言部分省略,
  该部分内容在正文中单独成章,标题改为绪论,用足够的文字叙述。
\end{enumerate}

\textcolor[rgb]{1.00,0.00,0.00}
{特别注意:是否如实引用前人结果反映的是学术道德问题,应明确写出同行相近的和已取得的成果,避免抄袭之嫌。} 